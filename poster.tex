\documentclass[final]{beamer}

\usepackage[orientation=portrait, size=a0, scale=1.4]{beamerposter}
\usepackage[absolute,overlay]{textpos}
\usepackage{xcolor}

\newlength{\sepwid}
\newlength{\onecolwid}
\newlength{\twocolwid}
\newlength{\threecolwid}
\setlength{\sepwid}{0.022\paperwidth}
\setlength{\onecolwid}{0.304\paperwidth}
\setlength{\twocolwid}{0.630\paperwidth}
\setlength{\threecolwid}{0.956\paperwidth}
\setlength{\topmargin}{-1in}
\usetheme{I6dv}
\usepackage{exscale}

\title{Scalable Hierarchical Schur Linear System Solver with Multilevel Parallelism on CUDA Enabled Clusters}
\author{Pochuan Wang, Weichung Wang}
\institute{Institute of Applied Mathematical Sciences, National Taiwan University}

\begin{document}
  \begin{frame}[t]
    \begin{columns}[t]
      \begin{column}{\sepwid}\end{column}
        % First column
        \begin{column}{\onecolwid}
        % Abstract block
        \begin{block}{Abstract}
          Sparse linear system solver is one of the core of scientific computing.
          As the scale of problem increases, to solve sparse linear systems efficiently
          is necessary.
          In recent computer architecture, the frequency of computation core is bounded
          by physical limitations, thus current design of computatation unit as CPU and
          GPU use multiple cores to improve the performance.
          Hierarchical Schur method expolits the block structure of multilevel nested
          dissection reordered sparse linear system and decompose the direct matrix
          factorization scheme into concurrent subproblems. In each subproblems we
          properly applied different techniques for lower level parallelism. Moreover
          by analyzing the computation cost of each subproblems, it is able to
          distribute the computation load to different resources to improve overall
          scalability.
        \end{block}
        % Introduction block
        \begin{block}{Introduction}
          \paragraph{\textbf{Why Solving $Ax=b$?}}
          \begin{itemize}
            \item Important kernel in scientific computing
            \item Widely used in real world applications
            \begin{itemize}
              \item Weather prediction
              \item Nano-structure optimization design 
              \item Renewable energy
              \item Biomedicine
            \end{itemize}
          \end{itemize}
          \paragraph{\textbf{How to solve $Ax=b$?}}
          \begin{itemize}
            \item Direct method: LU (Cholesky) factorization
          \end{itemize}
          \paragraph{\textbf{Difficulties}}
          \begin{itemize}
            \item High computation cost (BLAS Level 3)
            \item Hard to parallelize
          \end{itemize}
          \paragraph{\textbf{Our solutions}}
          \begin{itemize}
            \item Use GPU to accelerate BLAS-3 operations
            \item Improving parallel scalability by combining
            \begin{itemize}
              \item Nested Dissection Reordering
              \item Schur Complement Method
            \end{itemize}
          \end{itemize}
        \end{block}

        % Nested dissection block
          \begin{block}{Nested Dissection}
            \begin{itemize}
              \item A graph based algorithm for matrix reorder
              \item Apply to the graph which
              \begin{itemize}
                \item Vertices are correspond to matrix rows
                \item Edges are correspond to matrix non-zeros
              \end{itemize}
            \end{itemize}
            \includegraphics[0.8\onecolwid]{./ND_Example.pdf}
          \end{block}
      % End of first column
      \end{column}

      \begin{column}{\sepwid}\end{column}

      % Second column
      \begin{column}{\onecolwid}
        % Algorithm block
        \begin{block}{Hierarchical Schur Method}
          \paragraph{\textbf{Schur Complement Method}}
          \begin{itemize}
            \item Designed for matrices of this form
              \begin{equation*}
                A = \left[
                \begin{array}{ccc}
                  A_{11} &      0 & A_{13} \\
                       0 & A_{22} & A_{23} \\
                  A_{31} & A_{32} & A_{33}
                \end{array}
                \right]
              \end{equation*}
            \item By defining Schur’s complement $S$ as
              \begin{equation*}
                S = A_{33} - A_{31}A^{-1}_{11}A_{13} - A_{32}A^{-1}_{22}A_{23}
              \end{equation*}
            \item $A$ can be factorized as following
              $$
                A=
                \left[
                \begin{array}{ccc}
                                  I &                 0 & 0 \\
                                  0 &                 I & 0 \\
                  A_{31}A^{-1}_{11} & A_{32}A^{-1}_{22} & I
                \end{array}
                \right]
                \left[
                  \begin{array}{ccc}
                    A_{11} &      0 & 0 \\
                         0 & A_{22} & 0 \\
                         0 &      0 & S
                  \end{array}
                \right]
                \left[
                \begin{array}{ccc}
                  I & 0 & A^{-1}_{11}A_{13} \\
                  0 & I & A^{-1}_{22}A_{23} \\
                  0 & 0 &                 I
                \end{array}
                \right]
              $$
            \item $A^{-1}_{11}$ and $A^{-1}_{22}$ can be computed \textbf{in parallel}
          \end{itemize}

          \paragraph{\textbf{Hierarchical Schur Method}}


          \paragraph{\textbf{Algorithm by Example}}
        \end{block}
      \end{column}
      
      \begin{column}{\sepwid}\end{column}
      
      % Third column
      \begin{column}{\onecolwid}
        % Results block
        \begin{block}{Results}
          123
        \end{block}
        % Conclusion block
        \begin{block}{Conclusion}
          123
        \end{block}
      \end{column}

      \begin{column}{\sepwid}\end{column}
    \end{columns}
  \end{frame}
\end{document}
