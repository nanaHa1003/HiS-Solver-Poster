\documentclass[final]{beamer}

\usepackage[orientation=portrait, size=a0, scale=1.4]{beamerposter}
\usepackage[absolute,overlay]{textpos}
\usepackage{xcolor}

\newlength{\sepwid}
\newlength{\onecolwid}
\newlength{\twocolwid}
\newlength{\threecolwid}
\setlength{\sepwid}{0.016\paperwidth}
\setlength{\onecolwid}{0.304\paperwidth}
\setlength{\twocolwid}{0.624\paperwidth}
\setlength{\threecolwid}{0.944\paperwidth}
\setlength{\topmargin}{-1in}
\usetheme{I6dv}
\usepackage{exscale}

\title{Scalable Hierarchical Schur Linear System Solver with Multilevel Parallelism on CUDA Enabled Clusters}
\author{Pochuan Wang, Weichung Wang}
\institute{Institute of Applied Mathematical Sciences, National Taiwan University}

\begin{document}
  \begin{frame}[t]
    \begin{columns}[t]
      \begin{column}{\sepwid}\end{column}
        % First column
        \begin{column}{\onecolwid}
        % Abstract block
        \begin{block}{Abstract}
          Sparse linear system solver is one of the core of scientific computing.
          As the scale of problem increases, to solve sparse linear systems efficiently
          is necessary. In this research we developed a high performance direct linear
          solver for multicore cluster with GPU accelerator.
        \end{block}
        % Introduction block
        \begin{block}{Introduction}
          \paragraph{\textbf{Why Solving $Ax=b$?}}
          \begin{itemize}
            \item Important kernel in scientific computing
            \item Widely used in real world applications
            \begin{itemize}
              \item Weather prediction
              \item Nano-structure optimization design 
              \item Renewable energy
              \item Biomedicine
            \end{itemize}
          \end{itemize}
          \paragraph{\textbf{How to solve $Ax=b$?}}
          \begin{itemize}
            \item Direct method: LU (Cholesky) factorization
          \end{itemize}
          \paragraph{\textbf{Difficulties}}
          \begin{itemize}
            \item High computation cost (BLAS Level 3)
            \item Hard to parallelize
          \end{itemize}
          \paragraph{\textbf{Our solutions}}
          \begin{itemize}
            \item Use GPU to accelerate BLAS-3 operations
            \item Improving parallel scalability by combining
            \begin{itemize}
              \item Nested Dissection Reordering
              \item Schur Complement Method
            \end{itemize}
          \end{itemize}
        \end{block}

        % Nested dissection block
        \begin{block}{Nested Dissection}
          \begin{itemize}
            \item A graph based algorithm for matrix reorder
            \item Apply to the graph which
            \begin{itemize}
              \item Vertices are correspond to matrix rows
              \item Edges are correspond to matrix non-zeros
            \end{itemize}
          \end{itemize}
          \includegraphics{./ND_Example.pdf}
        \end{block}

        \begin{block}{Schur's Complement Method}
          \begin{itemize}
            \item Designed for matrices of this form
              \begin{equation*}
                A = \left[
                \begin{array}{ccc}
                  A_{11} &      0 & A_{13} \\
                       0 & A_{22} & A_{23} \\
                  A_{31} & A_{32} & A_{33}
                \end{array}
                \right]
              \end{equation*}
            \item By defining Schur’s complement $S$ as
              \begin{equation*}
                S = A_{33} - A_{31}A^{-1}_{11}A_{13} - A_{32}A^{-1}_{22}A_{23}
              \end{equation*}
            \item $A$ can be factorized as following
              \small
              \begin{equation*}
                A=
                \left[
                \begin{array}{ccc}
                                  I &                 0 & 0 \\
                                  0 &                 I & 0 \\
                  A_{31}A^{-1}_{11} & A_{32}A^{-1}_{22} & I
                \end{array}
                \right]
                \left[
                  \begin{array}{ccc}
                    A_{11} &      0 & 0 \\
                         0 & A_{22} & 0 \\
                         0 &      0 & S
                  \end{array}
                \right]
                \left[
                \begin{array}{ccc}
                  I & 0 & A^{-1}_{11}A_{13} \\
                  0 & I & A^{-1}_{22}A_{23} \\
                  0 & 0 &                 I
                \end{array}
                \right]
              \end{equation*}
              \normalsize
            \item $A^{-1}_{11}$ and $A^{-1}_{22}$ can be computed \textbf{in parallel}
          \end{itemize}
        \end{block}
      % End of first column
      \end{column}

      \begin{column}{\sepwid}\end{column}

      % Second column
      \begin{column}{\onecolwid}
        % Algorithm block
        \begin{block}{Hierarchical Schur Method}
          \paragraph{\textbf{Hierarchical Schur Method}}
          \begin{itemize}
            \item Considering a nested dissection reordered matrix as following, defining the terms as in the figure
          \end{itemize}
          \vskip1ex
          \includegraphics[width=0.95\onecolwid]{./ND_Blocks.pdf}
          \vskip1ex

          \paragraph{\textbf{Algorithm by Example}}
          \vskip1ex
          \ \ \ \includegraphics[width=0.9\onecolwid]{./Example_1.pdf} \\
          \vskip1ex
          \includegraphics[width=\onecolwid]{./Example_2.pdf} \\
          \vskip1ex
          \includegraphics[width=\onecolwid]{./Example_3.pdf} \\
          \vskip1ex
          \includegraphics[width=\onecolwid]{./Example_4.pdf} \\
          \vskip1ex
        \end{block}
      \end{column}
      
      \begin{column}{\sepwid}\end{column}
      
      % Third column
      \begin{column}{\onecolwid}
        % Implementation
        \begin{block}{Implementation}
          \paragraph{\textbf{Key Features}}
            \begin{itemize}
              \item Support Many-Core Heterogeneous Architecture
                \begin{itemize}
                  \item Multicore CPU with OpenMP
                  \item GPU Accelerator with NVIDIA CUDA
                  \item Inter Node with MPI
                \end{itemize}
              \item User Friendly Interfaces
              \begin{itemize}
                \item Header Only Library, No Installation Needed
                \item Full Documented
              \end{itemize}
              \item Good Extensibility and Maintainability
              \begin{itemize}
                \item Object Oriented Design with Modern C++11
                \item Using CppUnit Test Framework
              \end{itemize}
            \end{itemize}
        \end{block}

        % Results block
        \begin{block}{Results}
          
          \paragraph{\textbf{Test Environment}}
          \begin{itemize}
            \item Deep Computing Cluster
            \begin{itemize}
              \item Intel Xeon E5 2667 v2 3.3GHz * 2
              \item NVIDIA K40m GPU * 2
              \item Interconnection with InfiniBand
            \end{itemize}
          \end{itemize}

          \paragraph{\textbf{Test Problems}}
          \begin{itemize}
            \item Problems form Florida Matrix Collection
            \item Symmetric Positive Definite Matrices
            \item Dimensions of Matrices $>100,000$
          \end{itemize}

          \paragraph{\textbf{Factorization Time}} \\
          \begin{center}
            \includegraphics[width=0.8\onecolwid]{./Result_1.pdf}
          \end{center}

          \paragraph{\textbf{Overall Scalability}} \\
          \begin{center}
            \includegraphics[width=0.8\onecolwid]{./Result_2.pdf}
          \end{center}

        \end{block}
        % Conclusion block
        \begin{block}{Conclusions}
          \paragraph{\textbf{Conclusion}}
          \begin{itemize}
            \item Hierarchical Schur Method can solve large scale linear
                  systems efficiently.
            \item The overall scalability is bounded by the compute load
                  balancing in each compute unit. It can be further
                  improved by 
            \begin{enumerate}
              \item Better quality of nested dissection
              \item Dynamic task scheduling
            \end{enumerate}
          \end{itemize}
          \paragraph{\textbf{Future Works}}
          \begin{itemize}
            \item Non-symmetric matrix support
            \item Support for OpenCL and Xeon Phi Devices
          \end{itemize}
        \end{block}
      \end{column}

      \begin{column}{\sepwid}\end{column}
    \end{columns}
  \end{frame}
\end{document}
